% ---------
%  Compile with "pdflatex hw0".
% --------
%!TEX TS-program = pdflatex
%!TEX encoding = UTF-8 Unicode

\documentclass[11pt]{article}

\usepackage{jeffe,handout,graphicx,float}
\usepackage[utf8]{inputenc}		% Allow some non-ASCII Unicode in source
\usepackage{amsmath}
\newtheorem{lemma}{Lemma}
\usepackage{xcolor}
% =========================================================
%   Define common stuff for solution headers
% =========================================================
\pagenumbering{arabic}
\Class{CS/ECE 374 B}
\Semester{Fall 2019}
\Authors{2}
\AuthorOne{Jin Yucheng}{yucheng9}
\AuthorTwo{Fan Wenyan}{wenyanf2}
%\AuthorThree{Duncan Quagmire}{dquagmir}
%\Section{}

% =========================================================
\begin{document}

% ---------------------------------------------------------


\HomeworkHeader{4}{1}	% homework number, problem number

\begin{solution}
%These are, without exception, inappropriate inquiries, a phrase which here means “all the wrong questions”.  Here are the questions you should have asked instead:
%\begin{enumerate}[(a)]
%\item Why would someone say something was stolen when it was never theirs to begin with?
%\item How could someone who was missing be in two places at once?
%\item Why would someone destroy one building when they really wanted to destroy another?
%\end{enumerate}
%\begin{enumerate} [(a)]

\item (a) The exact solution is:
\begin{center}
$A(n) = n^2$, for any non-negative integer $n$
\end{center}
Assume $n$ is an arbitrary non-negative integer, for every non-negative integer $k < n$, we have $A(k) = k^2$, then:
\begin{itemize}
\item for $n = 0$, $A(0) = 0$ satisfies $A(n) = n^2$. 
\item for $n \geq 1$, by the inductive hypothesis, $A(n-1) = (n-1)^2$, such that:
\begin{center}
$A(n-1) + 2n - 1 = (n-1)^2 + 2n - 1 = n^2 - 2n + 1 + 2n - 1 = n^2 = A(n)$
\end{center}
\end{itemize}
\item Therefore, $A(n) = n^2$ is the exact solution for part (a).
\item (b) The exact solution is:
\begin{center}
$B(n) = \frac{1}{6} n^3 - \frac{1}{6} n$, for any non-negative integer $n$
\end{center}
Since for any non-negative integer $n > 1$, we have:
\begin{align*}
	B(n)
	& = B(n - 1) + {n \choose 2} \\
	& = B(n - 2) + {n - 1 \choose 2} + {n \choose 2} \\
	& = {2 \choose 2} +  ... + {n - 1 \choose 2} + {n \choose 2} \\
& \text{Because ${n \choose 2} = \frac{n!}{2!(n - 2)!} = \frac{(n-1)n}{2}$, we have,} \\
	& = \frac{1}{2} [1 \cdot 2 + 2 \cdot 3 + ... + (n - 2)(n - 1) + (n-1)n] \\
	& = \frac{1}{2} \cdot \frac{n(n-1)(n+1)}{3} \\
	& = \frac{1}{6} n^3 - \frac{1}{6} n
\end{align*}
Since $B(0) = 0$, $B(1) = B(0) + {1 \choose 2} = 0$ also satisfy $B(n) = \frac{1}{6} n^3 - \frac{1}{6} n$. Therefore, $B(n) = \frac{1}{6} n^3 - \frac{1}{6} n$ is the exact solution for part (b).
\\
\\
\\
\\
\item (c) We draw out the first few levels of the recursion tree for $C(n) = C(n/2) + C(n/3) + C(n/6) + n$:
\begin{figure}[H]
\centering
\includegraphics [width = 10cm]{WechatIMG1.jpeg}
\item
\end{figure}

\item Since $\frac{1}{2} + \frac{1}{3} + \frac{1}{6} = 1$, in all levels we have the equal sum, $n$, we immediately have $C(n) = O(n \cdot L) = O(nlogn)$ (see Note 1.7).
\item (d) We draw out the first few levels of the recursion tree for $D(n) = D(n/2) + D(n/3) + D(n/6) + n^2$:
\begin{figure}[H]
\centering
\includegraphics [width = 12cm]{WechatIMG3.png}
\item
\end{figure}
\item Since $\frac{1}{2^2} + \frac{1}{3^2} + \frac{1}{6^2} = \frac{7}{18}$, in all levels the sum decreases exponentially, which means every term is a constant factor smaller than the previous term, we immediately have $D(n) = O(n^2)$ (see Note 1.7).

\end{solution}


% ---------------------------------------------------------
% Change authors for all future solutions
\AuthorOne{Jin Yucheng}{yucheng9}
%\AuthorTwo{Friday Caliban}{fcaliban}
%\AuthorThree{Duncan Quagmire}{dquagmir}
\HomeworkHeader{4}{2}
\setcounter{page}{4}

\begin{solution}
The original algorithm is:
\item \# move "ndisks" from "src" to "dest" via "tmp" as temporary space
\item \qquad $1:$\textbf{ def} hanoi(ndisks, src, dest, tmp):
\item \qquad $2:$\qquad \textbf{ if} ndisks > 0:
\item \qquad $3:$\qquad \qquad hanoi(ndisks - 1, src, tmp, dest)
\item \qquad $4:$\qquad \qquad moveone(src, dest)
\item \qquad $5:$\qquad \qquad hanoi(ndisks - 1, tmp, dest, src)
\item \qquad $6:$\qquad \textbf{ else}: \# do nothing
\item (a) To make sure that moveone obeys the restriction that either the source or destination must be tower 0, after we move the largest disk from source to destination, we should put all the remaining disks back to source, so the modified algorithm is:
\item \# move "ndisks" from "src" to "dest" via "tmp" as temporary space
\item \qquad $1:$\textbf{ def} hanoi(ndisks, src, dest, tmp):
\item \qquad $2:$\qquad \textbf{ if} ndisks > 0:
\item \qquad $3:$\qquad \qquad hanoi(ndisks - 1, src, tmp, dest)
\item \qquad $4:$\qquad \qquad moveone(src, dest)
\item \qquad {\color{red} $5:$\qquad \qquad hanoi(ndisks - 1, tmp, src, dest)}
\item \qquad {\color{red} $6:$\qquad \qquad hanoi(ndisks - 1, src, dest, tmp)}
\item \qquad $7:$\qquad \textbf{ else}: \# do nothing \\
Where line 5 moves the remaining disks from "tmp" to "src" via "dest", which starts over the procedure to move disks from the source to destination, and what line 6 does is just moving the remaining disks from "src" to "dest" via "tmp".\\
Suppose the number of disks to move is $n$, the total number of calls to moveone is $F(n)$ if we need to move $n$ disks, therefore, we have: 
\begin{center}
$F(n) = F(n-1) + 1 + F(n-1) + F(n-1)$
\end{center}
Where three $F(n-1)$ represent the total number of calls to moveone in line 3, line 5, line 6, respectively. By solving:
\begin{center}
$F(n) = 3F(n-1) + 1$, $F(0) = 0, F(1) = 1$
\end{center}
We find that $F(n) = 3F(n-1) + 1 = 3F(n-2) + 3^1 + 3^0 = 3F(n-3) + 3^2 + 3^1 +3^0 = ...$ \\
So $F(n) = 3^0 + 3^1 + ... + 3^{n-1} = \frac {3^n - 1}{2}$, totally $\frac {3^n - 1}{2}$ calls to moveone are needed to move $n$ disks from the source to destination.
\\
\\
\\
\\
\item (b) The algorithm for moving $n$ disks from tower 0 to tower 1 is as follows:
\item \# move $n$ disks from tower 0 to tower 1 via tower 2 as temporary space
\item \# move "ndisks" from "src" to "dest" via "tmp" as temporary space
\item \qquad $\text{  }1:$\textbf{ def} hanoi(ndisks, src, dest, tmp):
\item \qquad $\text{  } 2:$\qquad \textbf{ if} ndisks > 0:
\item \qquad $\text{  } 3:$\qquad \qquad \textbf{\color{red}{ if} src = 2:}
\item \qquad $\text{  } 4:$\qquad \qquad \qquad{\color{red}moveall(2, tmp)}
\item \qquad $\text{  } 5:$\qquad \qquad \textbf{\color{red}  else:}
\item \qquad $\text{  } 6:$\qquad \qquad \qquad hanoi(ndisks - 1, src, tmp, dest)
\item \qquad $\text{  } 7:$\qquad \qquad moveone(src, dest)
\item \qquad $\text{  } 8:$\qquad \qquad \textbf{\color{red} { if} tmp = 2:}
\item \qquad $\text{  } 9:$\qquad \qquad \qquad {\color{red}moveall(2, dest)}
\item \qquad $10:$\qquad \qquad \textbf{\color{red} else:}
\item \qquad $11:$\qquad \qquad \qquad hanoi(ndisks - 1, tmp, dest, src)
\item \qquad $12:$\qquad \textbf{else}: \# do nothing \\
Compared to the original algorithm, we add two if and else statements to utilize moveall function. These two if and else statements mean to move the entire remaining stack of $n - 1$ disks from tower 2 to tower 0 or tower 1.\\
To calculate how many calls are needed to moveone and moveall, let us analyze two simple cases. 
\begin{itemize}
\item if we only have 1 disk, then we only need to use moveone once.
\item if we have 2 disks, then we need to use moveone tiwce and moveall once, the steps are indicated in the graph below. 
\end{itemize}
\begin{figure}[h]
\centering
\includegraphics [width = 11cm]{WechatIMG60.jpeg}
\end{figure}
\item Where the yellow arrows and circles show the number of moveone functions we call and blue arrows and circles show the number of moveall functions we call. Since the hanoi function is recursive, denote the number of moveall functions we need to call as $S(n)$,  the number of moveone functions we need to call as $F(n)$, we find that:
\begin{center}
$S(n) = \floor{n/2}$
\item $F(n) = F(n - 1) + F(n - 2) + 1$, for any integer $n > 2$, where $F(1) = 1, F(2) = 2$
\end{center}
\item By solving the linear recurrence relation of $F(n)$, we find that:
\begin{figure}[h]
\centering
\includegraphics [width = 11cm]{WechatIMG64.png}
\end{figure}
\end{solution}

\end{document}
